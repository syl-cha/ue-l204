\documentclass[12pt,a4paper,svgnames]{simplereport}

% Packages
\usepackage{boxes}
\usepackage{minted}
\usepackage{xcolor}
\usepackage{hyperref}
\usepackage{caption}
\usepackage{pdfpages}
\usepackage{setspace}
\captionsetup{
    font={small,it},           % Taille small et italique pour toute la légende
    labelfont={bf,color=black}, % Label en gras et en bleu
    textfont={color=black!60},    % Texte de la légende en noir
    labelsep=endash,           % Séparateur avec tiret cadratin
    format=plain,              % Format simple
    justification=justified,   % Justification du texte
    width=0.8\linewidth        % Largeur à 80% de la ligne (ajustez selon besoin)
}
\geometry{margin=2.5cm}
\graphicspath{{images/}}

% Définition des couleurs
\definecolor{backColor}{RGB}{248,248,248}
\definecolor{subColor}{RGB}{70,130,180}

% Configuration de tcolorbox
\tcbuselibrary{minted,skins,breakable}

% Environnement pour le code JavaScript
\newtcblisting{javascriptcode}{
    listing engine=minted,
    minted style=colorful,
    minted language=javascript,
    minted options={
        fontsize=\small,
        linenos,
        numbersep=3mm,
        firstnumber=1,
        breaklines,breakanywhere,
        breakbytoken
    },
    colback=backColor,
    colframe=subColor,
    listing only,
    left=5mm,
    enhanced,
    sharp corners,
    boxrule=0pt,
    breakable,
    overlay={
        \begin{tcbclipinterior}
            \fill[subColor!30] (frame.south west) rectangle ([xshift=5mm]frame.north west);
        \end{tcbclipinterior}
    }
}

% Configuration des liens
\hypersetup{
    colorlinks=true,
    linkcolor=orange,
    urlcolor=orange,
    citecolor=orange
}


\newcommand{\codetext}[1]{%
\textcolor{Crimson}{\texttt{#1}}
}

\usepackage{titletoc}

% Espacement standard pour les pointillés (ajuste l'espace pour le numéro de page)
\contentsmargin{2.5em} 

% Format pour \part (ex: I, II, III)
% Le texte de la partie (ex: "Introduction") sera en gras
\titlecontents{part}
  [0em]                                      % Indentation à gauche
  {\addvspace{10pt}\bfseries}                % Espace AVANT (réduit) + format texte
  {\textcolor{orange}{\thecontentslabel}\quad} % Format du NUMÉRO (orange + espace)
  {\bfseries}                                  % Format si pas de numéro (gras)
  {\ \titlerule*[.5pc]{.}\ \contentspage}     % Pointillés et page
  [\addvspace{2pt}]                          % Espace APRES (réduit)

% Format pour \section (ex: 1, 2, 3)
% Le texte de la section sera en police normale
\titlecontents{section}
  [1.5em]                                    % Indentation (plus grande que part)
  {\addvspace{2pt}}                         % Espace AVANT (très réduit)
  {\textcolor{orange}{\thecontentslabel}\quad} % Format du NUMÉRO (orange + espace)
  {}                                         % Format si pas de numéro
  {\ \titlerule*[.5pc]{.}\ \contentspage}     % Pointillés et page
  []                                         % Pas d'espace APRES

% Format pour \subsection (ex: 1.1, 1.2)
\titlecontents{subsection}
  [3.0em]                                    % Indentation (plus grande que section)
  {\addvspace{1pt}}                         % Espace AVANT (minimal)
  {\textcolor{orange}{\thecontentslabel}\quad} % Format du NUMÉRO (orange + espace)
  {}                                         % Format si pas de numéro
  {\ \titlerule*[.5pc]{.}\ \contentspage}     % Pointillés et page
  []                                         % Pas d'espace APRES

% Format pour \subsubsection (ex: 1.2.1, 1.2.2)
\titlecontents{subsubsection}
  [4.5em]                                    % Indentation (plus grande que subsection)
  {\addvspace{1pt}}                         % Espace AVANT (minimal)
  {\textcolor{orange}{\thecontentslabel}\quad} % Format du NUMÉRO (orange + espace)
  {}                                         % Format si pas de numéro
  {\ \titlerule*[.5pc]{.}\ \contentspage}     % Pointillés et page
  []                                         % Pas d'espace APRES

  % Titre

\title{UE-L204 MINI-PROJET\\
\vspace*{2cm}
{\titlefont \fontsize{40pt}{36pt}\selectfont Rapport intermédiaire}\\
\vspace*{2cm}
% \includegraphics[width=0.9\linewidth]{hero_section.png}}
\author{Sylvain \textsl{Chambon},\\%
Jade \textsl{Faroux},\\%
Jeanne \textsl{Salvadori},\\ %
Zoé \textsl{Van De Moortele}}
\date{\today}}
\begin{document}

\maketitle

\section{La méthode de travail}
\subsection{Outils utilisés}
\begin{itemize}
  \item Teams pour la communication et le partage de fichiers ;
  \item GitHub pour la mise en commun du code sur un \href{https://github.com/syl-cha/ue-l204}{dépôt}
        et pouvoir mieux gérer les modifications via le système de branche ;
  \item Forum de groupe de l'UE : communication et retour sur l'avancement du projet ;
  \item Word pour une rédaction commune et \LaTeX pour la finalisation des rapports ;
  \item un système d'intelligence artificielle générationnelle (Gemini) pour générer les entrées dans les tables.
\end{itemize}

\subsection{Étude préliminaire}

\subsection{Scénario}

Créer un site universitaire avec différents niveaux d’accès : étudiant et professeur, éventuellement un administrateur.

Les professeurs peuvent :
\begin{itemize}
  \item Créer, modifier, supprimer des cours.
  \item Voir la liste d’étudiant inscrit.
  \item Gérer les inscriptions (accepter/refuser les étudiants dans leurs cours).
  \item Modifier leur profil (informations personnelles).
\end{itemize}

Les étudiants peuvent :
\begin{itemize}
  \item Consulter l’ensemble des cours disponibles.
  \item Rechercher des cours (nom, professeur, domaine).
  \item S’inscrire à des cours.
  \item Modifier leur profil (informations personnelles).
\end{itemize}

Si assez de temps, mettre en place un système de note (prof donne des notes et les étudiants peuvent y accéder), et d’emploi du temps.

\subsection{Objectifs adaptés à notre scénario}

\subsubsection{Page de connexion}

\begin{itemize}
  \item Formulaire d’entrée où l’utilisateur entre son login et mot de passe.
  \item Système vérifie si cet utilisateur existe dans la BDD.
  \item Le mot de passe doit être crypté.
  \item Si correct, accès au portail étudiant ou professeur.
  \item Si incorrect, message d’erreur.
  \item Changement de mot de passe lors de la première connexion (sécurité).
\end{itemize}

\subsubsection{Page de recherche}

\begin{itemize}
  \item Champ de recherche de cours (par nom, professeur ou domaine).
  \item Doit interroger la BDD avec les critères de recherche.
  \item Afficher les résultats sous forme de tableau, ou message d’erreur si aucun résultat.
\end{itemize}

\subsubsection{Page d’ajout de contenu}

\begin{itemize}
  \item Les professeurs ont une page dédiée pour ajouter, modifier ou supprimer leur cours.
  \item Les professeurs et étudiants peuvent modifier leurs informations personnelles (mail, tel, mot de passe...).
  \item Toutes ces modifications doivent être enregistrées dans la BDD.
\end{itemize}

\subsubsection{Vérification des données}

Avant d’enregistrer quoi que ce soit dans la BDD, tests de sécurité à effectuer systématiquement :
\begin{itemize}
  \item Avant d’enregistrer quoi que ce soit dans la BDD, tests de sécurité à effectuer systématiquement :
  \item Vérifier que les champs obligatoires soient remplis
  \item Les formats des entrées doivent être corrects
  \item Pas de caractères dangereux (balise html)
  \item Messages d’erreur si une de ces vérifications est non conforme
\end{itemize}

\section{Avancement du projet}

\subsection{Base de données}

Nous avons modélisé notre base de données dans \href{https://app.diagrams.net/}{diagrams.net}.
Nous avons pensé à définir deux utilisateurs pour notre base de données : l'enseignant et l'étudiant.
En structurant notre base, nous nous sommes aperçu que ces deux entités partageaient beaucoup 
d'attributs donc nous avons préférer définir l'entité plus générale d'\texttt{utilisateur}
et définir les entités \texttt{enseignant} et \texttt{etudiant} comme des héritages de l'entité
\texttt{utilisateur} avec certains attributs spécifiques.

Dans cette base de données, nous aurons évidemment aussi une entité \texttt{cours}.
\medskip

Nous aurons enfin des tables de liaison :
\begin{itemize}
  \item une table permettant d'associer des enseignants avec des cours (entité \texttt{enseigne}) ;
  \item un autre permttant d'associer des étudiants avec des cours (entité \texttt{etudiant}) ;
  \item une dernière permettant de définir des contraintes sur les inscriptions dans certains cours (entité \texttt{prerequis}).
\end{itemize}

\subsubsection{Entité \texttt{utilisateur}}
L'entité \texttt{utilisateur} aura les attributs (voir requête \no{\ref{sql:utilisateur}}) :
\begin{itemize}
  \item \texttt{login} : identifiant de connexion ;
  \item \texttt{mot\_de\_passe} : mot de passe pour la connextion : 
  il sera encrypté par l'application (ce n'est pas de la responsabilité de la BBD) ;
  \item \texttt{mot\_de\_passe\_provisoire} : booléen servant de drapeau pour savoir si le mot de passe initial a été changé ou pas ;
  \item \texttt{nom}, \texttt{prenom}, \texttt{email} : pour renseigner des élément lié à la personne et son contact ;
  \item \texttt{role} : trois rôles sont défini ici pour pouvoir tester les droits relatifs à chaque utilisateur de la BDD ;
  \item \texttt{date\_creation} pour stocker la date à laquelle l'utilisateur a été saisi dans la BDD ;
  \item \texttt{actif} drapeau pour connaître si la personne est encore en activité dans la faculté.
\end{itemize}

\begin{sql}[title={Création de la table \texttt{utilisateur}},label={sql:utilisateur}]
CREATE TABLE utilisateur (
  id INT PRIMARY KEY AUTO_INCREMENT,
  login VARCHAR(255) UNIQUE NOT NULL,
  mot_de_passe VARCHAR(255) NOT NULL,
  mot_de_passe_provisoire BOOLEAN DEFAULT TRUE,
  nom VARCHAR(100) NOT NULL,
  prenom VARCHAR(100) NOT NULL,
  email VARCHAR(255) UNIQUE NOT NULL,
  role ENUM('enseignant', 'etudiant', 'admin') NOT NULL,
  date_creation TIMESTAMP DEFAULT CURRENT_TIMESTAMP,
  actif BOOLEAN DEFAULT TRUE
);
\end{sql}

On se rend compte que tous ces attributs serons partagés à la fois par les enseignants et par les étudiants.

Un ajout d'un utilisateur dans la BDD peut donc se déclarer ainsi (voir exemple de requête \no{\ref{sql:utilisateur:creation}})

\begin{sql}[title={Création d'un utilisateur dans la table},label={sql:utilisateur:creation}]
(1, 'turing', '<mot-de-passe>', 'Turing', 'Alan', 'alan.turing@univ.fr', 'enseignant'),
\end{sql}

\subsubsection{Entité \texttt{enseignant}}

L'entité \texttt{enseignant} hérite de de l'entité \texttt{utilisateur} avec quelques spécificités :
\begin{itemize}
  \item \texttt{bureau} : localisation de la salle de travail de l'enseignant ;
  \item \texttt{telephone} : numéro ;
  \item \texttt{specialite} : domaine d'expertise ;
  \item \texttt{statut} : sous quel titre l'enseignant a-t-il été embauché.
\end{itemize}

On constate la présence d'une clé étrangère afin de lier l'entité \texttt{enseignant}
avec une entité \texttt{utilisateur} existante (héritage).

\begin{sql}[title={Création de la table \texttt{enseignant}},label={sql:enseignant}]
CREATE TABLE enseignant (
  id INT PRIMARY KEY AUTO_INCREMENT,
  utilisateur_id INT UNIQUE NOT NULL,
  bureau VARCHAR(50),
  telephone VARCHAR(20),
  specialite VARCHAR(255),
  statut ENUM('titulaire', 'vacataire', 'contractuel') DEFAULT 'titulaire',
  FOREIGN KEY (utilisateur_id) REFERENCES utilisateur(id) ON DELETE CASCADE
);
\end{sql}

On ajoutera un enseignant ainsi dans la base (voir exemple de requête \no{\ref{sql:enseignant:creation}})

\begin{sql}[title={Création d'un enseignant dans la table},label={sql:enseignant:creation}]
INSERT INTO enseignant (utilisateur_id, bureau, telephone, specialite, statut) VALUES
(1, 'B101', '0102030401', 'Intelligence Artificielle', 'titulaire');
\end{sql}



\subsubsection{Entité \texttt{etudiant}}

L'entité \texttt{etudiant} hérite de de l'entité \texttt{utilisateur} avec quelques spécificités :
\begin{itemize}
  \item le classique \texttt{numero\_etudiant}, comme référence nationale ;
  \item \texttt{niveau} qui référence si l'étudiant est en licence ou master et en quelle année ;
  \item \texttt{date\_inscription} qui peut-être de la \texttt{date\_creation} si l'étudiant n'a pas validé ses frais de scolarité par exemple.
\end{itemize}
On constate ici aussi la présence d'une clé étrangère afin de lier l'entité \texttt{étudiant}
avec une entité \texttt{utilisateur} existante (héritage, comme pour l'entité \texttt{enseignant}).

\begin{sql}[title={Création de la table \texttt{etudiant}},label={sql:etudiant}]
CREATE TABLE etudiant (
  id INT PRIMARY KEY AUTO_INCREMENT,
  utilisateur_id INT UNIQUE NOT NULL,
  numero_etudiant VARCHAR(20) UNIQUE NOT NULL,
  niveau ENUM('L1', 'L2', 'L3', 'M1', 'M2') NOT NULL,
  date_inscription DATE NOT NULL,
  FOREIGN KEY (utilisateur_id) REFERENCES utilisateur(id) ON DELETE CASCADE
);
\end{sql}

Pour ajouter un étudiant dans la base, on pourra procéder ainsi (voir exemple de requête \no{\ref{sql:etudiant:creation}})

\begin{sql}[title={Création d'un étudiant dans la table},label={sql:etudiant:creation}]
INSERT INTO etudiant (utilisateur_id, numero_etudiant, niveau, date_inscription) VALUES
(12, '20250001', 'L3', '2024-09-01');
\end{sql}

\subsubsection{Entité \texttt{cours}}

\begin{sql}[title={Création de la table \texttt{etudiant}},label={sql:cours}]
CREATE TABLE etudiant (
  id INT PRIMARY KEY AUTO_INCREMENT,
  utilisateur_id INT UNIQUE NOT NULL,
  numero_etudiant VARCHAR(20) UNIQUE NOT NULL,
  niveau ENUM('L1', 'L2', 'L3', 'M1', 'M2') NOT NULL,
  date_inscription DATE NOT NULL,
  FOREIGN KEY (utilisateur_id) REFERENCES utilisateur(id) ON DELETE CASCADE
);
\end{sql}


\section{Difficultés rencontrées / pas encore résolues}

\section{Projections pour la 2\ieme{} semaine}
\end{document}